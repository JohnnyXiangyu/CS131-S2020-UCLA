\documentclass[letterpaper,twocolumn,10pt]{article}
\usepackage{usenix2019_v3}

% to be able to draw some self-contained figs
\usepackage{tikz}
\usepackage{amsmath}
\usepackage{listings}
\microtypecontext{spacing=nonfrench}


% inlined bib file
\usepackage{filecontents}

%-------------------------------------------------------------------------------
\begin{document}
%-------------------------------------------------------------------------------

%don't want date printed
\date{}

% make title bold and 14 pt font (Latex default is non-bold, 16 pt)
\title{\Large \bf Dart Assessment Report For Garage Garner}

%for single author (just remove % characters)
\author{
{\rm Xiangyu Wan}\\
UCLA
% copy the following lines to add more authors
% \and
% {\rm Name}\\
%Name Institution
} % end author

\maketitle


%-------------------------------------------------------------------------------
\section{Overall}
%-------------------------------------------------------------------------------

To enhence user experience during garage sales, Garage Garner app is seeking a serverless solution for better responsiveness 
and less server bandwidth consumption.
For our project, combination of Dart, Flutter, and TF lite is a good fit, as far as I am concerned.
While TF lite grants the app ability to complete its task completely natively,
Dart, as a popular native app programming language, is easy to use and provides good performance, 
generality and security, as well as a reasonable level of flexibility, all of which are essential to this app.

%-------------------------------------------------------------------------------
\section{Serverless Machine Learning}
%-------------------------------------------------------------------------------

We begin the discussion with TF lite, while most of the paper will center around Dart.
To begin with, it's simple to use.
The API is simple to use and extremely concise, and such simplicity would allow an image recognition program to be coded in merely 80 lines of code
while fully operational on a raspberry pi (reference 2).
Simplicity of this core computation module allows the app to be developed and deployed with more ease.

More importantly, TF lite is specially optimized for mobile devices.
Compared with its standard version, the lite branch consumes less processor power and disk space.
TF lite models are trimmed in size to fit into the disk size of mobile devices, which are often not so spacious.
It's also very efficient. 
While cell phones cannot compare with servers in terms of computational power, TensorFlow lite still enables machine learning algorithms to run natively
while no help from the server is needed.
This independence from server is essential for Garage Garner, as it affects user experience greatly by introducing extra lag and network requirement.

The tradeoff, however, is accuracy.
Compared to the standard comparts, TensorFlow Lite models generally have less accuracy due to the optimization done on them.
The decrease in accuracy, however, won't out shine the efficiency we gain from these optimizations.
Moreover, the nature of Garage Garner actually allows some level of inaccuracy.
For example, we can allow users to manually modify predicted data after a photo has been processed,
as a more comfortable way of getting manual labelling from clients.
These labelling data will then be sent to the server for later models' training.

Continuous training and deployment of models, as mentioned in the previous example, is also well supported by TF lite.
While TF lite differ in implementation from the standard version, it use models directly convertable from standard TensorFlow.
Therefore, training process can keep running on the server side, and pushing newer versions of ML models to clients on a regular time base.

%-------------------------------------------------------------------------------
\section{Ease}
%-------------------------------------------------------------------------------

The first and most obvious advantage of Dart is its ease and simplicity.
Dart is an imperative and object oriented programming language, with C like syntax and a strict static typing system.
At this point it appears to C++ or Java, both of which are imperative, OOP languages.
The benefit of a simplified C-like syntax is the familarity a programmer feels when he or she first encounters Dart.
They would only need to learn the Dart-specific features, therefore, before they can hands-on Dart development.
On top of that, nontheless, Dart also support functional language features, including lambda expressions, which have been intensely used in frontend
development in other languages as well.
These extra language features makes Dart more powerful than its imperative ancesters who it share the same syntax structure with.

Dart is also easier than C++ or Java in terms of writing code, due to the type inference system it implements.
In addition to its static type system, the compiler also infers type from a variable or parameter's usage.
The programmer, in turn, can omit some of the type labels from the code.
This extra level of flexibility allows Dart code to be written in more concise and readable ways.
As a brief conclusion of syntactical simplicity, Dart merges advantages from C, Java, and Ocaml.
Python, however, differ greatly with Dart as dynamic typing system is used in Python.
It will be further discussed in reliability section.

Besides the ease to learn and get used to Dart syntax, the widget system in Flutter-Dart projects also immensely eases the hassle of creating and managing a graphical user interface.
It works in a similar manner as how a React.js component works, that each widget, corresponding to a GUI element, is controlled by its set of states.
The change of states dedicates when and how a widget should be updated.
Just like how React.js is known for its simplicity, the widget system allows code manipulating GUI to have simple and easy implementation.
Compared to other languages learnt this quarter, Dart has absolute advantage over all of them, 
as non of Java, Python, or OCaml natively supports creation and management of graphic user interface.

%-------------------------------------------------------------------------------
\section{Performance}
%-------------------------------------------------------------------------------

Compared to some of other native application development languages, like React native, Dart has a natural advantage, that it can be compiled into ARM machine code.
For React native, the system still need to translate javascript into machine code to execute, while Dart application can directly run on the processor.
Also, static typing improves a program's efficiency, like how Java or C beats Python in efficiency besides the difference between compiled and script langauges.

Yet beyond that hardcore difference down to assembly level, Dart also has good support to both single-thread and multi-thread concurrency.
For single-thread applications, there's the future system, which is Dart's implementation of the same idea Asyncio or Node.js's promise system bear.
It's also an event loop that yields tasks waiting for IO automatically to allow multiple IO-heavy tasks to happen at once.
For example, in Dart, a call to fetch response from a url returns a future. 
Programmer can then attach .then() methods or its like to the future object, telling the application what to do with the actual response.

Another strategy is to multithread the program, which is builtin feature of Dart.
In fact, single-thread program in Dart is but a special case of its multithreading system, as execution is organized into isolates in Dart, 
and each 'isolate' runs on a separate thread.
Therefore, moving a single-thread program, or a prototype, to multithreaded version for a speed up is easy to implement in Dart.
Furthermore, Dart isolates execute with totally isolated memory, and all of inter-thread communication is done through messages, another mechanism implemented in Dart.
This way, programmers are freed from the exhausting process of designing synchronization mechanisms, only focusing on optimization.

As shown in the Python server project, asynchronous I/O or multi-threading are essential to implement a server if one would like the server to run on single core while processing client data concurrently.
In this case, however, it's also important for our mobile application, since this client side application will constantly be running machine learning predictions
in the 'background', while doing other things including sending and receiving data, such as labels done by the client, to and from the server.
GUI, in addition, also runs parallel with everything else.
This need of parallelism is perfectly met by the concurrency support from Dart.
For example, a separate isolate could be created specifically for machine learning algorithms from the thread where widgets are being managed;
and within the main thread, the program can use asynchronous procedures to perform IO operations such as sending user's label to the server, or receive messages from the server.
With the consistently increasing power of modern day mobile processors, Dart applications can shine with its simple implementation of powerful concurrency mechanism;
as in the previous example, the machine learning model can run on the accelerator, while in-app functionalities run on different cores of the processor.
The responsiveness of Garage Garner is thus protected from the potential lag caused by internet instability or the length process of recognizing an image.

%-------------------------------------------------------------------------------
\section{Reliability}
%-------------------------------------------------------------------------------

Dart also has good reliability assurance.
As mentioned above, Dart is static typed, which means it's less likely to have unchaught errors due to a mismatch in argument types.
Dynamic typed languages, despite the immense flexibility gained by getting rid of the type symbol, sometimes require programmers to handle type checking manually.
Facing this decision, Dart goes for the stability and reliability provided by static typing system.

A related criteria of reliability to typing is null types, which is a latest release feature. 
In many languages, including Python and Java, null types contribute to most of the runtime errors.
Dart, however, has its null-safe system, by which such null-value deferencing error no longer needs special treatment.

Beyond the scope of single-threaded reliability, Dart is also not prone to error caused by erroneous inter thread communication.
As mentioned in performance section, in Dart each thread, or isolation, has its own memory, and different isolates can't get access to another's memory as easily as in Java.
Therefore, memory is protected from race conditions even without extra definition and declerations like what one must include to use JMM in Java.

In short, Dart is a reliable language system in terms of typing, null types and race condtions. 
The fact that it's backed by Google and open source community also adds to that conclusion.

%-------------------------------------------------------------------------------
\section{Generality and Flexibility}
%-------------------------------------------------------------------------------

All that reliability concerns come at a price of limiting flexibility of Dart. 
Static typing and forced memory isolation do eliminate some possibilities of how a program can be written, but their benefits can be more influential.
Besides how reliability concerns affect it, however, Dart is still reasonably flexible for the portability it has.
There's no limitation of Dart on any of its supported platforms.
To transport a program to a differnt platform, say from Android to IOS, copying the code over to a different compiler would do the work.
Python and Java offer portability in a same manner, but both of them have system-specific API that can't be ported from one platform to another.
For example, the API of read() and write() is specific to Unix family operating systems.
Dart beats them in portability.

The use of lambda expressions also add to Dart's flexibility. 
By allowing lambda as well as high-order functions to be written in Dart, part of a Dart application, or even all of it, can be written in a functional way.
The magnitude of how functional or imperative depends on the programmer.
For example, one can write a list processing module entirely in functional style, yet write the caller in imperative style.
More often though, it's going to be a mixture of the 2 in most modules.
It's very difficult, however, to do the same thing in either Python or Java: to create a full procedure inside a statement of code, and treat that as a normal variable, passing it to functions or use that as return value.
Yet in times of doing iterative jobs, such feature can really help to simplify the code.

A similar area where Dart also does well is generality.
Dart has a powerful generics type system.
One can use it to create generic functions, classes, or collection objects.
When used in classes and functions, it's pretty much the same way as how generic types are used in Java that they are treated just as normla type identifiers.
The generic type is only defined after a call to the method has been made to instantiate the generic prototype.
Just like Java, progammer can define restrictions on generic types enforcing certain relationships between its arguments.
Moreover, generic collections store their type information at runtime, so that a procedure can dynamically find what type a collection, say list, if defined to contain.
Compared to Python's solution to generic type, which is a dynamic typing system implementing the duck typing ideology, Dart's generic type is yet less powerful.
They are still confined to a static type system, just added with some variation.

Another Dart's effort towards greater generality is its Extension feature, added in the most recent release.
Through extension, Dart program can modify even builtin types, giving them new methods or operators.
What's more, these new methods also support generic type variables.
With this feature, programmer won't need to create a whole lot of wrappers for specific need to entend existing data types.
It can be done by just adding extensions to the builtin type, with all the powerful genric support for normal functions.

%-------------------------------------------------------------------------------
\section{Conclusion}
%-------------------------------------------------------------------------------

In conclustion, Dart suits the job well.
Compared with Java, Python, and OCaml, it does well in terms of difficulty, performance, reliability, and generality.
And even though its flexibility is limited by the need for performance and reliability, it portability and functional style support still allows Dart to be considered flexible.
TF lite, the machine learning framework selected to work with Flutter and Dart, also grants enough power to a single mobile device to carry out machine learning based prediction individually.

\newpage
%-------------------------------------------------------------------------------
\section{References}
%-------------------------------------------------------------------------------

https://towardsdatascience.com/a-beginners-introduction-to-tensorflow-lite-924320deed5

https://github.com/tensorflow/examples/blob/master/lite/examples/image\_classification/raspberry\_pi/classify\_picamera.py

https://dart.dev/guides

https://medium.com/dartlang/dart-2-7-a3710ec54e97

%-------------------------------------------------------------------------------
\bibliographystyle{plain}
% \bibliography{\jobname}

%%%%%%%%%%%%%%%%%%%%%%%%%%%%%%%%%%%%%%%%%%%%%%%%%%%%%%%%%%%%%%%%%%%%%%%%%%%%%%%%
\end{document}
%%%%%%%%%%%%%%%%%%%%%%%%%%%%%%%%%%%%%%%%%%%%%%%%%%%%%%%%%%%%%%%%%%%%%%%%%%%%%%%%

%%  LocalWords:  endnotes includegraphics fread ptr nobj noindent
%%  LocalWords:  pdflatex acks